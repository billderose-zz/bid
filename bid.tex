\documentclass[11pt, oneside]{article}     % use "amsart" instead of "article" for AMSLaTeX format
\usepackage{geometry}                		% See geometry.pdf to learn the layout options. There are lots.
\geometry{letterpaper}                   		% ... or a4paper or a5paper or ... 
%\geometry{landscape}                		% Activate for for rotated page geometry
%\usepackage[parfill]{parskip}    		% Activate to begin paragraphs with an empty line rather than an indent
\usepackage{ulem}
\usepackage{graphicx}				% Use pdf, png, jpg, or epsß with pdflatex; use eps in DVI mode
\usepackage{amsmath}
								% TeX will automatically convert eps --> pdf in pdflatex		
\usepackage{amssymb}
\usepackage[colorlinks=true,urlcolor=blue]{hyperref}
\title{The Bitcoiner's Dilemma}
\author{Bill DeRose}
%\date{}							% Activate to display a given date or no date
\usepackage{Sweave}
\begin{document}
\input{bid-concordance}
\maketitle
I'd like to present a simplified analysis of the bidding that took place for the 
USMS's bitcoin this past Friday. We model the scenario as a two player
zero-sum game between me (I would really like to own those coins) 
and a single other bidder. Both players may choose from the following set of 
moves $$S =\{\mbox{ABOVE MARKET}, \mbox{AT MARKET}, \mbox{BELOW MARKET}\}.$$

We will assume that if both players submit a bid at market price, a coin
is flipped to decide the winner. The situation becomes complicated when the 
players submit matching bids of either both above or both below market price.
One could even imagine a matrix game within a matrix
game where once we know both bids are in the same range the players enter a new
game where the moves become $5\%, 10\%$, or $15\%$ above/below market price.
For simplicity's sake, if both players submit matching bids, a coin
is flipped to decide the winner.
The matrix game is then 
\begin{align*}
G = \bordermatrix{
      &   AM & @M & BM\cr
      AM & 0.5 & 1 & 1\cr
    	@M & 0 & 0.5 & 1\cr
			BM & 0 & 0 & 0.5\cr}
\end{align*}
where the payoffs are to me, ``player one". In this case my moves correspond to 
selecting rows so 
my maximin strategy is $x = \begin{bmatrix}1 & 0 & 0 \end{bmatrix}^t$. 
There are a few ways to arrive at this answer and we will begin with
the simplest before diving a little further into the math behind it.


If we don't want to dirty our hands with the computation once we've set
up our game we can use a solver readily available from
\href{http://www.math.ucla.edu/~tom/gamesolve.html}{\uline{UCLA}}. Just plug in the
matrix and let it spit out the answer.

However, if we are so inclined, we can find the optimal strategy ourselves
by solving a linear program. Let $x' = \begin{bmatrix}x_1 & x_2 & x_3\end{bmatrix}^t$ 
be the probability vector that describes my bidding strategy. 
That is, $x_1$ is  the probability that I bid $above$ the market price, $x_2$ 
is the probability that I bid $at$ the market price, and $x_3$ is the 
probability that I bid $below$ the market price. To find our optimal 
strategy, we must solve the following LP
\begin{align*}
\text{Maximize~~~~}&\min\{0.5x_1,~x_1 +0.5x_2,~x_1 + x_2+0.5x_3\}\\
\text{Subject to~~~~}&x_1 + x_2 + x_3 = 1\\
&x_1, x_2, x_3 \ge 0
\end{align*}
We wish to maximize our minimum payoff, which depends on the move
our opponent (``player two") selects. The functions whose minimum we wish
to maximize correspond to 
player two's decision to bid above, at, or below market price, respectively.
However, the $\min$ function is not linear so we must transform the LP 
to an equivalent one that a linear solver can handle
\begin{align*}
\text{Maximize~~~~}&x_4\\
\text{Subject to~~~~}&0.5x_1 \ge x_4 \\
&x_1 +0.5x_2 \ge x_4\\
&x_1 + x_2+0.5x_3 \ge x_4\\
&x_1 + x_2 + x_3 = 1\\
&x_1, x_2, x_3 \ge 0
\end{align*}
Since there are no constraints on $x_4$, maximizing it ``pushes" its value
up until $x_4$ is exactly equal to 
$\min\{0.5x_1,~x_1 +0.5x_2,~x_1 + x_2+0.5x_3\}$. Hence we have achieved our goal
of maximizing our minimum payoff.
The following R code solves the LP for us
\begin{Schunk}
\begin{Sinput}
> library(lpSolve)
> obj <- c(0, 0, 0, 1)
> con <- matrix(c(0.5, 0, 0, -1,
+                 1, 0.5, 0, -1,
+                 1, 1, 0.5, -1,
+                 1, 1, 1, 0), byrow = T, nrow = 4)
> rhs <- c(0, 0, 0, 1)
> dir <- c(">=", ">=", ">=", "=")
> lp("max",obj, con, dir, rhs)
\end{Sinput}
\begin{Soutput}
Success: the objective function is 0.5 
\end{Soutput}
\begin{Sinput}
> lp("max",obj, con, dir, rhs)$solution
\end{Sinput}
\begin{Soutput}
[1] 1.0 0.0 0.0 0.5
\end{Soutput}
\end{Schunk}
Thus the optimal values are $x_1 = 1, x_2 = 0, x_3 = 0, 
x_4 = 0.5$ and our optimal strategy is to always (with probability 1) bid 
above the market price.


An even easier way to arrive at this solution is by using 
\href{http://en.wikipedia.org/wiki/Strategic_dominance}{\uline{dominance}}
to eliminate possible moves. We can throw out
moves two and three because I am guaranteed to always do better (i.e. my payoff
is always greater) if I play move one than if I were to play 
move two or three. Hence, the game I see may be simplified to
\begin{align*}
G' = \bordermatrix{
      &   AM & @M & BM\cr
      AM & 0.5 & 1 & 1}
\end{align*}
and again my maximin strategy is $x = \begin{bmatrix}1 & 0 & 0 \end{bmatrix}^t$.


The reader is encouraged to solve the LP a third way: graphically. Start by plotting 
$0.5x_1,~x_1 + 0.5x_2$, and $x_1 + x_2 + 0.5x_3$ against one another
and go from there.


\end{document}  
